% T I T L E   P A G E
% -------------------
% Last updated June 14, 2017, by Stephen Carr, IST-Client Services
% The title page is counted as page `i' but we need to suppress the
% page number. Also, we don't want any headers or footers.
\pagestyle{empty}
\pagenumbering{roman}

% The contents of the title page are specified in the "titlepage"
% environment.
\begin{titlepage}
        \begin{center}
        \vspace*{1.0cm}

        \Huge
        {\bf Refresh Strategies in Continuous Active Learning }

        \vspace*{1.0cm}

        \normalsize
        by \\

        \vspace*{1.0cm}

        \Large
        Nimesh Ghelani \\

        \vspace*{3.0cm}

        \normalsize
        A thesis \\
        presented to the University of Waterloo \\ 
        in fulfillment of the \\
        thesis requirement for the degree of \\
        Master in Mathematics \\
        in \\
        Computer Science \\

        \vspace*{2.0cm}

        Waterloo, Ontario, Canada, 2018 \\

        \vspace*{1.0cm}

        \copyright\ Nimesh Ghelani 2018 \\
        \end{center}
\end{titlepage}

% The rest of the front pages should contain no headers and be numbered using Roman numerals starting with `ii'
\pagestyle{plain}
\setcounter{page}{2}

\cleardoublepage % Ends the current page and causes all figures and tables that have so far appeared in the input to be printed.
% In a two-sided printing style, it also makes the next page a right-hand (odd-numbered) page, producing a blank page if necessary.

 
% D E C L A R A T I O N   P A G E
% -------------------------------
  % The following is a sample Delaration Page as provided by the GSO
  % December 13th, 2006.  It is designed for an electronic thesis.
  \noindent
I hereby declare that I am the sole author of this thesis. This is a true copy of the thesis, including any required final revisions, as accepted by my examiners.

  \bigskip
  
  \noindent
I understand that my thesis may be made electronically available to the public.

\cleardoublepage

% A B S T R A C T
% ---------------

\begin{center}\textbf{Abstract}\end{center}
High recall information retrieval is crucial to tasks such as electronic
discovery and systematic review. \textit{Continuous Active Learning} (CAL) is a
technique where a human assessor works in loop with a machine learning model; the
model presents a set of documents likely to be relevant and the assessor
provides relevance feedback. Our focus in this thesis is on one particular aspect
of CAL:  \textit{refreshing}, which is a crucial and recurring event in the CAL
process.  During a \textit{refresh}, the machine learning model is trained with
the relevance judgments and a new list of likely-to-be-relevant documents is
produced for the assessor to judge. It is also computationally the most
expensive step in CAL. In this thesis, we investigate the effects of the default
and alternative refresh strategies on the effectiveness and efficiency of CAL.
We find that more frequent refreshes can significantly reduce the human effort
required to achieve certain recall. For moderately sized datasets, the high
computation cost of frequent refreshes can be reduced through a careful
implementation. For dealing with resource constraints and large datasets, we
propose alternative refresh strategies which provide the benefits of frequent
refreshes at a lower computation cost.
In this thesis, we also discuss the design of a modern implementation of the CAL
algorithm which is efficient and extensible. Our implementation can be used as a
research tool as well as for practical applications.

\cleardoublepage

% A C K N O W L E D G E M E N T S
% -------------------------------

\begin{center}\textbf{Acknowledgements}\end{center}

I would like to thank my supervisor Dr. Mark D. Smucker for his effort and guidance
throughout my time here. Mark gave me the freedom and support to work on
problems I was interested in, for which I am very grateful.

I thank Dr. Gordon V. Cormack and Dr. Maura R. Grossman for agreeing to be the readers
of this thesis. Their course on ``High Stakes Information Retrieval" played a
significant role in defining my research interests. Gordon's feedback and
suggestions were crucial for the content in this thesis.

I had the privilege of being part of an amazing research group; Mustafa
Abualsaud and Haotian Zhang were a joy to work with. I am also very thankful to
the members of the Data System Group: Dr. Jimmy Lin, Amine Mhedhbi, Angshuman
Ghosh, Chathura Kankanamge, Royal Sequeira, Shahin Rahbariasl, Siddhartha Sahu
and Vineet John; who enriched my time in University of Waterloo in many
different ways.

Finally, I would like to thank my parents and brother for their constant
support. Their presence, despite being thousands of miles away, was a great
source of encouragement.

\cleardoublepage

% D E D I C A T I O N
% -------------------

\begin{center}\textbf{Dedication}\end{center}

\centerline{This thesis is dedicated to everyone who directly or indirectly, made
it possible.}

\cleardoublepage

% T A B L E   O F   C O N T E N T S
% ---------------------------------
\renewcommand\contentsname{Table of Contents}
\tableofcontents
\cleardoublepage
\phantomsection    % allows hyperref to link to the correct page

% L I S T   O F   T A B L E S
% ---------------------------
\addcontentsline{toc}{chapter}{List of Tables}
\listoftables
\cleardoublepage
\phantomsection		% allows hyperref to link to the correct page

% L I S T   O F   F I G U R E S
% -----------------------------
\addcontentsline{toc}{chapter}{List of Figures}
\listoffigures
\cleardoublepage
\phantomsection		% allows hyperref to link to the correct page

% % GLOSSARIES (Lists of definitions, abbreviations, symbols, etc. provided by the glossaries-extra package)
% % -----------------------------
% \printglossaries
% \cleardoublepage
% \phantomsection		% allows hyperref to link to the correct page

% Change page numbering back to Arabic numerals
\pagenumbering{arabic}

