% Parent problem, approaches, CAL and its history
High Recall Information Retrieval is crucial to tasks such as (but not limited
to) electronic discovery and systematic review. In high recall information
retrieval, the goal is to find all or nearly all relevant documents using
minimal human effort. This is in contrast to popular web-based search engines
like Google which are built to deliver high precision (but low recall) results
to its users.

Various approaches addressing the problem of high recall information retrieval
under various applications
exist~\cite{li2014req,hogan2010automation,cormack2014evaluation}. One such set
of approaches is called Technical Assisted Review (TAR).  In a TAR process, a
computer system uses judgments made by human assessors to classify documents as
either relevant or non-relevant. TAR methods outperform manual review in legal
eDiscovery by reducing the cost spent on human
assessors~\cite{grossman2010technology,roitblat2010document}.   Continuous
active learning (CAL)~\cite{cormack2014evaluation,cormack2015autonomy} is a TAR
protocol where a machine learning algorithm suggests most likely relevant
documents for human assessment and continuously incorporates relevance feedback to improve
its understanding of the search task.  In a previous study, Cormack and
Grossman~\cite{cormack2014evaluation} showed that CAL outperforms other TAR
protocols on review tasks from actual legal matters and TREC 2009 Legal Track.
The Total Recall track in TREC 2015 and 2016 evaluated different systems under a
simulated TAR setting~\cite{grossman2016trec,roegiest2015trec}.  Baseline Model
Implementation (BMI) based on CAL was used as the baseline in these tracks. BMI
implements the AutoTAR algorithm~\cite{cormack2015autonomy}.  None of the
participating systems were able to consistently outperform the baseline.

In this thesis, we modify and extend the AutoTAR CAL algorithm. We isolate an
important step of the algorithm and call it \textit{refreshing}.  During a
\textit{refresh}, the relevance judgments from the assessor are used to train a
new classifier. This classifier generates an ordered list of documents most
likely to be relevant, which is later processed by the assessor. Apart from
being a crucial factor in the effectiveness of the retrieval algorithm, this
step also has a high computation cost because it involves training a classifier
and computing relevance likelihood scores for potentially all the documents in
the corpus. A \textit{refresh strategy} determines when and how to perform the
refresh.

We propose different refresh strategies and compare their effect on the
performance of CAL. By studying various refresh strategies, we aim to:
\begin{itemize} \item Improve effectiveness of CAL; specifically its capability
to achieve higher recall with lesser effort \item Improve computational
efficiency of CAL; so that it is responsive and  feasible in a production
environment.  \end{itemize}

In the AutoTAR algorithm, refreshing is done after a certain number of documents
are assessed. This number increases exponentially over time. We find we can make CAL
more effective by refreshing after every assessment. However, refresh is an
expensive operation. Frequent refreshing can severely impact the running time
and usability of a CAL system. We also design and evaluate
alternative refresh strategies which have lower computation cost and achieve similar
effectiveness.  These strategies can be considered when dealing with resource
constraints or large datasets.

The work described in this thesis required a suitable implementation of CAL.
Another contribution of this thesis is a modern and efficient implementation of
CAL which can support aggressive refresh strategies. Our tool is designed to be
used both as a research tool and in real world applications.  


\section{Thesis Organization}

The organization of the remainder of this thesis is described below.

In Chapter~\ref{chap:rel}, we introduce prerequisites to understanding the
subsequent chapters of this thesis. We discuss the Continuous Active Learning
algorithm and define the concept of \textit{refresh} and \textit{refresh
strategy}. We then review related work approaching similar problems and work
which can be potentially applied to the problem addressed in this thesis.

In Chapter~\ref{chap:implementation}, we describe the design and features of
the implementation using which all the experiments in this thesis were
performed.

In Chapter~\ref{chap:dataset}, we discuss the design of our experiments, along
with the dataset and evaluation metrics used.

In Chapter~\ref{chap:refresh}, we define and explain various refresh strategies.  In
Chapter~\ref{chap:results}, we evaluate and compare the performance of these
refresh strategies.

In Chapter~\ref{chap:conclusion}, we discuss the conclusions of our work
and the future work addressing various practical applications.
