% dataset and stats
\label{chap:dataset}

\section{Dataset}
Our experiments were performed using multiple \texttt{athome} test collections.

The \texttt{athome1} and \texttt{athome4} collections were used in the TREC 2015 and 2016 Total Recall
Track~\cite{roegiest2015trec,grossman2016trec}, respectively. Both consist of
290,099 documents, which are redacted emails from Jeb Bush's eight year tenure
as the Governor of Florida. \texttt{athome1} has 10 topics with an average 4398
documents labelled as relevant per topic. \texttt{athome4} has 34 topics with an
average of 1059.44 documents labelled as relevant per topic. The topics for
\texttt{athome1} and \texttt{athome4} are described in
Table~\ref{tab:topics_athome1} and~\ref{tab:topics_athome4}, respectively.

\texttt{athome2} and \texttt{athome3} collections were used in the TREC 2015
Total Recall Track~\cite{roegiest2015trec} as well as in the TREC 2015 Dynamic
Domain Track~\cite{yang2015overview}. \texttt{athome2}, or the \textit{Illicit
Goods Collection}, consists of 465,147 web forum threads from
BlackHatWorld\footnote{\url{https://www.blackhatworld.com/}} and
Hack Forums\footnote{\url{https://hackforums.net/}}. It has 10 topics with an
average of 2000.5 documents labelled as relevant per topic. \texttt{athome3}, or
the \textit{Local Politics Collection}, consists of 902,434 news article from
various sources in United States and Canada. It has 10 topics with an
average of 642.9 documents labelled as relevant per topic. The topics for
\texttt{athome2} and \texttt{athome3} are described in
Table~\ref{tab:topics_athome2} and~\ref{tab:topics_athome3}, respectively.

The \texttt{athome} test collections are suitable for evaluating high recall
systems due to the extensive judgments available for their topics. They were
used in the Total Recall tracks, which evaluated participating systems similar
to how we evaluate different refresh strategies. Moreover, some of the
\texttt{athome} collections were assessed (\texttt{athome1}) or re-assessed
(\texttt{athome2} and \texttt{athome3}) using Continuous Active Learning based
methods. These factors make the \texttt{athome} collections a suitable choice of
dataset to conduct our experiments on.

% gain curve

\section{Evaluation}
\label{sec:eval}

We describe the performance metrics we used to compare the
effectiveness and efficiency of refresh strategies in the following sections.
For each refresh strategy, these metrics were computed at the dataset level by
averaging the topic-level metric for each topic in that dataset. These
dataset-level metrics were further averaged (macro average of topic-level
metrics) to compute the average metric for a refresh strategy.

\subsubsection{Effectiveness}
To measure the effectiveness of various refresh strategies, we ran a CAL simulation for
each topic and strategy. We can compare different strategies based on the recall
values they achieved at various values of effort. For each topic, the recall is
defined as

\begin{equation}
    Recall = \frac{\text{No. of relevant documents found by the system}}{\text{Total no. of
    relevant documents in the corpus}}
\end{equation}

Using absolute values of review effort can cause an imbalance among the recall
values of individual topics. This is due to uneven number of relevant documents
present across different topics (see
Table~\ref{tab:topics_athome1}-\ref{tab:topics_athome4}). Instead of
absolute effort, we use normalized effort $E_{norm}$ for analyzing results.
$E_{norm}$ is calculated by dividing the absolute review effort by the total
number of relevant documents.

The gain curve is a widely used method to evaluate high recall retrieval
systems~\cite{roegiest2015trec,grossman2016trec,grossman2011overview}. It is a
plot of recall as a function of assessed documents. We use average recall gain
curves to compare the effectiveness of different refresh strategies.  A gain
curve for a topic is a plot of recall (y-axis) against the normalized review
effort ($E_{norm} = \frac{E}{R}$), where $E$ is the number of judgments made
since the beginning of the simulation and $R$ is the total number of relevant
documents for that topic. We get the dataset-level average gain curve by
averaging the recall values over all its topics. We further average all the
dataset-level recall values to obtain the average gain curve for a refresh
strategy.

For the sake of readability, we also report certain points of interest from the
gain curve in a tabular format. Specifically, we compare different refresh strategies based on
their recall values when $E_{norm} \in \{1,1.5,2\}$.  We also report the effort
required to reach $75\%$ recall.

To measure the statistical significance of performance difference between two
strategies, we use the Student's paired t-test on a performance metric (such as
recall at $E_{norm} \in \{1,1.5,2\})$. When comparing two refresh strategies, we
perform statistical significance tests separately for each dataset.

\subsubsection{Efficiency}
A combination of experiment conditions (such as refresh strategy and its
parameters) and a topic is referred as a ``task''. We use review effort as the
stopping criteria for each task (step 11 of Algorithm ~\ref{alg.cal}). One
judgment is equal to one unit of reviewer's effort. The CAL process for a task
stops whenever the number of judgments processed by the system is equal to the
maximum effort. The maximum normalized review effort ($E_{norm}$) in our
efficiency experiments was set to $2$.

To measure the computational efficiency, we record the running time of the
simulation. Different refresh strategies are compared based on the running time
of the simulation averaged over all the topics of a dataset. The dataset-level
average running times are averaged to obtain the average running time of a
strategy.


\section{Runtime Environment}

We used the command line interface of the CAL implementation discussed in
Chapter~\ref{chap:implementation} to run our experiments. We ran our experiments
on a google cloud instance with 11 GB memory and twelve vCPUs (2.00 GHz Intel Xeon
Processor). We ran a CAL process for each strategy and parameter setting
sequentially. A CAL process simulated a maximum of twelve parallel tasks. For
each task, we restricted the scoring of documents to a single thread.

\section{Secondary Experiments}
\label{sec:secondary}

We performed few additional experiments to support certain intuitions behind the
design of refresh strategies discussed in Section~\ref{sec:async} and
Section~\ref{sec:partial}. The questions these experiments were designed to
answer are:
\begin{itemize}
    \item If a judgment causes a refresh, a new review queue is populated.
        In the new queue, what is the position of the first unjudged document
        from the previous queue?
    \item To what extent does the classifier's notion of relevance change across
        various number of judgments?
\end{itemize}

The simulations were run across the 10
\texttt{athome1} topics and the reported metrics were averaged over them. The
maximum review effort was set to $500$.



\begin{table}[h]
\centering
\caption{List of \texttt{athome1} topics}
\label{tab:topics_athome1}
\begin{tabular}{|c|c|c|}
\hline
\textbf{Topic ID} & \textbf{Description} & \textbf{Relevant Documents} \\ \hline \hline
athome100 & School and Preschool Funding & 4542 \\ \hline
athome101 & Judicial Selection & 5836 \\ \hline
athome102 & Capital Punishment & 1624 \\ \hline
athome103 & Manatee Protection & 5725 \\ \hline
athome104 & New medical schools & 227 \\ \hline
athome105 & Affirmative Action & 3635 \\ \hline
athome106 & Terri Schiavo & 17135 \\ \hline
athome107 & Tort Reform & 2375 \\ \hline
athome108 & Manatee County & 2375 \\ \hline
athome109 & Scarlet Letter Law & 506 \\ \hline
\end{tabular}
\end{table}

\begin{table}[h]
\centering
\caption{List of \texttt{athome2} topics}
\label{tab:topics_athome2}
\begin{tabular}{|c|c|c|}
\hline
\textbf{Topic ID} & \textbf{Description} & \textbf{Relevant Documents} \\ \hline \hline
athome2052 & paying for amazon book reviews                   & 265 \\ \hline
athome2108 & CAPTCHA Services                                 & 661 \\ \hline
athome2129 & Facebook Accounts                                & 589 \\ \hline
athome2130 & Surely bitcoins can be used                      & 2299 \\ \hline
athome2134 & paypal accounts                                  & 252 \\ \hline
athome2158 & \begin{tabular}[x]{@{}c@{}}
Using TOR for anonymous \\browsing on the internet
\end{tabular} & 1256 \\ \hline
athome2225 & Rootkits                                         & 182 \\ \hline
athome2322 & Web Scraping                                     & 9517 \\ \hline
athome2333 & article spinner spinning                         & 4805 \\ \hline
athome2461 & Offshore Host Sites                              & 179 \\ \hline
\end{tabular}
\end{table}

\begin{table}[h]
\centering
\caption{List of \texttt{athome3} topics}
\label{tab:topics_athome3}
\begin{tabular}{|c|c|c|}
\hline
\textbf{Topic ID} & \textbf{Description} & \textbf{Relevant Documents} \\ \hline \hline
athome3089 & pickton murders                     & 255 \\ \hline
athome3133 & pacific gateway                     & 113 \\ \hline
athome3226 & traffic enforcement cameras         & 2094 \\ \hline
athome3290 & rooster turkey chicken nuisance     & 26 \\ \hline
athome3357 & occupy vancouver                    & 629 \\ \hline
athome3378 & rob mckenna gubernatorial candidate & 66 \\ \hline
athome3423 & rob ford cut the waist              & 76 \\ \hline
athome3431 & kingston mills lock murders         & 1111 \\ \hline
athome3481 & fracking                            & 2036 \\ \hline
athome3484 & paul and cathy lee martin           & 23 \\ \hline
\end{tabular}
\end{table}

\begin{table}[h]
\centering
\caption{List of \texttt{athome4} topics}
\label{tab:topics_athome4}
\begin{tabular}{|c|c|c|}
\hline
\textbf{Topic ID} & \textbf{Description} & \textbf{Relevant Documents} \\ \hline \hline
athome401 & Summer Olympics                   & 229 \\ \hline
athome402 & Space                             & 647 \\ \hline
athome403 & Bottled Water                     & 1091 \\ \hline
athome404 & Eminent Domain                    & 548 \\ \hline
athome405 & ewt Gingrich                      & 122 \\ \hline
athome406 & Felon Disenfranchisement          & 131 \\ \hline
athome407 & Faith-Based Initiatives           & 1587 \\ \hline
athome408 & Invasive Species                  & 116 \\ \hline
athome409 & Climate Change                    & 206 \\ \hline
athome410 & Condominiums                      & 1354 \\ \hline
athome411 & "Stand Your Ground"               & 89 \\ \hline
athome412 & 2000 Recount                      & 1422 \\ \hline
athome413 & James V. Crosby                   & 552 \\ \hline
athome414 & Medicaid Reform                   & 841 \\ \hline
athome415 & George W. Bush                    & 12106 \\ \hline
athome416 & Marketing                         & 1452 \\ \hline
athome417 & Movie Gallery                     & 5952 \\ \hline
athome418 & War Preparations                  & 187 \\ \hline
athome419 & Lost Foster Child Rilya Wilson    & 1989 \\ \hline
athome420 & Billboards                        & 742 \\ \hline
athome421 & Traffic Cameras                   & 21 \\ \hline
athome422 & Non-Resident Aliens (NRA)         & 33 \\ \hline
athome423 & National Rifle Association (NRA)  & 286 \\ \hline
athome424 & Gulf Drilling                     & 500 \\ \hline
athome425 & Civil Rights Act of 2003          & 718 \\ \hline
athome426 & Jeffrey Goldhagen                 & 121 \\ \hline
athome427 & Slot Machines                     & 246 \\ \hline
athome428 & New Stadiums and Arenas           & 466 \\ \hline
athome429 & Cuban Child, Elian Gonzales       & 828 \\ \hline
athome430 & Restraints and Helmets            & 999 \\ \hline
athome431 & Agency Credit Ratings             & 150 \\ \hline
athome432 & Gay Adoption                      & 140 \\ \hline
athome433 & Abstinence                        & 112 \\ \hline
athome434 & Bacardi Trademark Lobbying        & 38 \\ \hline
\end{tabular}
\end{table}

