% dataset and stats
\label{sec.dataeval}
We used the Athome1 test collection from the TREC 2015 Total Recall
track~\cite{roegiest2015trec}. The collection has 290,099 documents and 10
topics. Each topic has an average of 4398 documents labelled as relevant.

% gain curve

To measure the effectiveness, we ran a CAL simulation for each topic and refresh
strategy until $E_{max}$ number of judgments were made. $E_{max}$ is also known
as the max review effort. In our experiments, $E_{max}$ was equal to $2\times R$ where
$R$ is the total number of relevant documents for that topic. We use average
recall gain curves to compare the effectiveness of different refresh strategies.
A gain curve for a topic is a plot of recall (y-axis) against the normalized
review effort ($E_{norm} = \frac{E}{R}$), where $E$ is the number of judgments made since
the beginning of the simulation. We get the average gain curve by averaging the
recall over the 10 topics. For the sake of readability and space, we excluded
those plots from this paper. Instead, we report certain points of interest from
the plots as a table. Specifically, we compare different refresh strategies based
on their recall values when $E_{norm} \in \{1,1.5,2\}$. We also report the
effort required to reach $75\%$ recall for each refresh strategy.

% timings
To measure the computational efficiency, we fixed $E_{max}=10000$ and ran the
simulation across four 2.20GHz CPUs (Step 7 of Algorithm~\ref{alg.cal} is the
only parallelized step) for each topic and refresh strategy. Different refresh
strategies are compared based on the running time of the simulation averaged
over the 10 topics.
