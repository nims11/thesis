\label{sec.cal}
% CAL algorithm (assuming an arbitrary refresh strategy)
A general version of the AutoTAR CAL algorithm is described in
Algorithm~\ref{alg.cal}.  The choice of refresh strategy can control when to
perform a refresh (step 10), as well as the behaviour of training (step 4) and
scoring (step 5). In this paper, we simulate human assessors using a set of
existing relevance judgments (Step 8). Unlabelled documents are considered
non-relevant during the simulation.

\begin{algorithm}[]
Construct a seed document whose content is the topic description \\
Label the seed document as relevant and add it to the training set \\
Add 100 random documents from the collection, temporarily labeled as ``not
relevant'' \\
Train a Logistic Regression classifier using the training set \\
Remove the random documents from the training set added in step 3 \\
Flush the review queue \\
Using the classifier, order documents by their relevance scores and put them
into a review queue \\ Review a document from the review queue, coding it as
``relevant'' or ``not relevant'' \\
Add the document to the training set \\
Repeat steps 8-9 until a refresh is needed (defined by the refresh strategy) \\
Repeat steps 3-10 until some stopping condition is met.
\caption{AutoTAR CAL Algorithm (assuming an arbitrary refresh strategy). A refresh
strategy can control behaviour of steps 4, 7 and 10}
\label{alg.cal}
\end{algorithm}

Documents are represented as a vector of unigram tf-idf features which are used
for training the classifier and calculating relevance likelihood scores. BMI
AutoTAR uses sofia-ml\footnote{https://code.google.com/archive/p/sofia-ml/}
\cite{sculley2010combined} to train a logistic regression classifier. It uses
the \textit{logreg-pegasos} learner with $100000$ iterations of \textit{roc}
sampling. A training iteration involves randomly sampling a relevant and a
non-relevant document from the training set, computing the loss and adjusting
the classifier weights accordingly. The classifier weights are used to calculate
relevance likelihood score for any document.

The BMI tool provided by the TREC Total Recall organizers is written in BASH.
For efficiency and extensibility, we implemented Algorithm~\ref{alg.cal} in
C++~[Blinded]. Our implementation organizes all the document feature vectors in
the memory to eliminate disk accesses and enable faster operations. New refresh
strategies can be easily implemented and tested using this tool. The tool
provides a command line interface to run simulations and a HTTP API for use in
real world applications.
